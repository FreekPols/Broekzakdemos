
% ======================
% === PAGE LAYOUT ===
% ======================
\usepackage[top=2cm, bottom=2cm, left=2cm, right=2cm]{geometry}  % Adjust margins here for pages

% ================================
% === FONT FAMILY & TEXT SIZE ===
% ================================
% --- CHANGE MAIN FONT HERE ---
% The default font is lmodern (Latin Modern), but you can change it to something else
% Example: \usepackage{times} for Times Roman font, or \usepackage{helvet} for Helvetica
% Options: times, palatino, mathptmx, helvet, etc.
\usepackage{lmodern}  % Default font is Latin Modern. Change here if you'd like a different font!

% =======================
% === ENCODING ===
% =======================
\usepackage[utf8]{inputenc}  % Input encoding for UTF-8 characters
\usepackage[T1]{fontenc}  % Font encoding to use special characters (like accents)

% ===============================
% === LINE SPACING (Optional) ===
% ===============================
% Uncomment the following lines if you want to change line spacing for the entire document.
% Default is single-spaced. Uncomment to change to one-and-a-half spacing or double spacing.
% \usepackage{setspace}
% \onehalfspacing  % Or you can use \doublespacing for double spacing.

% ==============================
% === SECTION/CHAPTER STYLING ===
% ==============================
\usepackage{titlesec}

% ======================
% === CUSTOMIZING CHAPTER TITLE FORMAT ===
% ======================
% You can customize the chapter title appearance with the following options.
% The \titleformat command has this structure: 
% \titleformat{\chapter}[display]{<Font Style>}{<Chapter Number>}{<Space Between Number & Title>}{<Title Formatting>}
%
% 0. [display]: This defines how the chapter title is displayed. Other options include [block], [runin], etc.
% 1. Font Style: This defines the style of the chapter title text.
%    Common options: 
%    - \bfseries (bold)
%    - \itshape (italic)
%    - Font sizes: \Huge, \huge, \LARGE, \large, \small, etc.
%    Example: \bfseries\Huge makes the chapter title bold and large.
%
% 2. Chapter Number: This controls how the chapter number looks.
%    - Chapter inserts the word "Chapter". You can remove or replace this with any text (e.g., "Part").
%    - \thechapter inserts the chapter number (e.g., "1", "2", etc.).
%    - You can modify alignment: \filleft (right-align), \centering (center-align), \flushleft (left-align).
%    - You can adjust the vertical space before the chapter number using \vspace{-1em}.
%    Example: \vspace{-1em}\flushleft\huge\chaptername~\thechapter makes the chapter number huge and aligned to the left.
%
% 3. Space Between Number & Title: This defines the space between the chapter number and title.
%    The value is given in units like ex, pt, etc. Example: 1ex, 2ex, etc.
%    Adjust this to increase or decrease the gap between the number and title.
%
% 4. Title Formatting: This controls the formatting of the chapter title itself.
%    You can set the alignment (e.g., \centering, \flushleft, \filleft) and the font size (e.g., \Huge, \LARGE).
%    Example: \filleft\Huge right-aligns and uses a huge font size for the chapter title.

%You  can also change the chapter formatting by for example inserting this at the top of your chapter:
% \renewcommand{\rmdefault}{ptm}  % e.g., Times Roman for this chapter
% \chapter{Introduction}
% or 
% {\large
% \chapter{Introduction}
% This whole paragraph is in a larger font.
% }


\titleformat{\chapter}[display]
  {\bfseries\Huge}             % Font style
  {Chapter \thechapter}  % Chapter number and alignment
  {20pt}                        % Space between number and title
  {\filleft\Huge}              % Title font size and alignment

% --- CHANGE SECTION STYLE HERE ---
% Controls how sections look (just below chapters).
% \titleformat{\section}
%   {\large\bfseries}     % Bold and large size for section titles
%   {\thesection}         % The section number (e.g., 1.1, 1.2)
%   {1em}                 % Space between section number and title
%   {}                    % Title formatting
\titleformat{\section}
  {\Large\bfseries}          % Font style for sections (Bold and large)
  {\thesection}              % Section number formatting
  {1em}                      % Space between number and title
  {}

% --- CHANGE SUBSECTION STYLE HERE ---
% Similar to section but for subsections (the next level).
% \titleformat{\subsection}
%   {\normalsize\bfseries}    % Smaller font, still bold
%   {\thesubsection}          % Subsection number (e.g., 1.1.1)
%   {1em}                     % Space between number and title
%   {}                        % Title formatting
\titleformat{\subsection}
  {\normalsize\bfseries}     % Font style for subsections (Bold and normal size)
  {\thesubsection}           % Subsection number
  {1em}                      % Space between number and title
  {}

% ======================
% === MATH PACKAGES ===
% ======================
\usepackage{amsmath}  % For math symbols and environments
\usepackage{amssymb}  % For additional math symbols

% ======================
% === GRAPHICS ===
% ======================
\usepackage{graphicx}  % For including images
\usepackage{caption}   % For figure captions

% ======================
% === COLORS & LINKS ===
% ======================
\usepackage{xcolor} 
\usepackage{hyperref}  % For hyperlinks in the document
\hypersetup{
  colorlinks,           % Enables colored links
  linkcolor={black},    % Set color of internal links (e.g., table of contents)
  citecolor={black},    % Set color of citation links
  urlcolor={black}      % Set color of external links
}

% ======================
% === CITATIONS ===
% ======================
\usepackage{natbib}  % For handling citations
\bibliographystyle{unsrtnat}  % Set your preferred citation style (unsrtnat, abbrv, etc.)

% ======================
% === LIST FORMATTING ===
% ======================
\usepackage{enumitem}
\setlist[itemize]{noitemsep, topsep=0pt}  % More compact bullet points (reduces space)

% ======================
% === HEADER/FOOTER ===
% ======================
% Uncomment and customize if you want custom header/footer styles
% \usepackage{fancyhdr}
% \pagestyle{fancy}
% \fancyhead[L]{}
% \fancyhead[C]{}
% \fancyhead[R]{}
% \fancyfoot[C]{\thepage}  % Page number centered in footer

